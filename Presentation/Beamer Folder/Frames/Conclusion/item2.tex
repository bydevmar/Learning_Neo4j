\begin{frame}{Conclusion }
  \begin{block}{Cypher Query Language}
    Nous avons exploré la syntaxe et les structures de requêtes de Cypher, avec des exemples concrets pour interroger et manipuler des données graphiques de manière efficace.
  \end{block}
  
  \begin{block}{Architecture et Performances}
    L'architecture de Neo4j, incluant le stockage de données, l'indexation et les techniques d'optimisation des performances, ainsi que des considérations sur le clustering, la réplication et la haute disponibilité, ont été abordées.
  \end{block}
\end{frame}