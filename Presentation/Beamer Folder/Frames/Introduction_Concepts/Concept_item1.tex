\begin{frame}{Nœuds, relations}
  \begin{block}{Nœuds}
    Les nœuds sont les entités fondamentales dans une base de données Neo4j. Ils représentent des entités individuelles telles que des personnes, des lieux ou des objets, et sont souvent utilisés pour stocker des données de manière structurée.
  \end{block}
  \begin{block}{Relations}
    Les relations sont des liens entre les nœuds qui représentent les connexions ou les interactions entre les entités. Elles permettent de modéliser les relations complexes et les dépendances dans les données.
  \end{block}
\end{frame}

\begin{frame}{Propriétés et Étiquettes}
  \begin{block}{Propriétés}
    Les propriétés sont des paires clé-valeur associées aux nœuds et aux relations, permettant de stocker des informations supplémentaires pertinentes. Elles offrent une flexibilité pour ajouter des détails spécifiques à chaque entité ou relation.
  \end{block}
  \begin{block}{Étiquettes}
    Les étiquettes sont utilisées pour regrouper des nœuds similaires, facilitant ainsi les requêtes et l'analyse des données. Elles permettent de catégoriser les nœuds en fonction de leurs caractéristiques communes.
  \end{block}
\end{frame}
