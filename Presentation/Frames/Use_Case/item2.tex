\begin{frame}{Scénarios où Neo4j excelle}
  \begin{block}{Gestion de la chaîne d'approvisionnement}
    Neo4j aide les entreprises à optimiser leur chaîne d'approvisionnement en visualisant et en analysant les relations entre les fournisseurs, les produits et les processus.
  \end{block}
  \begin{block}{Gestion des identités et des accès}
    Avec Neo4j, les organisations peuvent gérer efficacement les identités et les accès en comprenant les relations entre les utilisateurs, les groupes et les permissions.
  \end{block}
  \begin{block}{Nomenclatures}
    Neo4j est utilisé pour gérer les nomenclatures, permettant aux entreprises de suivre les composants et les produits à travers des structures complexes.
  \end{block}
\end{frame}