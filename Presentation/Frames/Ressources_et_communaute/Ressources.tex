\section{Ressources et communauté}

\begin{frame}{Ressources}
  \begin{block}{Documentation et tutoriels}
    La documentation officielle de Neo4j est disponible sur le site de Neo4j, offrant des guides complets pour débuter, des manuels de référence pour Cypher, et des tutoriels pour divers cas d'utilisation.\footnote{Liens : \url{https://neo4j.com/docs/}, \url{https://neo4j.com/docs/cypher-manual/current/}}\footnote{Liens : \url{https://neo4j.com/graphacademy/}}\footnote{Liens : \url{https://neo4j.com/docs/}} 
  \end{block}
  \begin{block}{Obtenir de l'aide}
    Pour obtenir de l'aide technique, vous pouvez consulter la base de connaissances de Neo4j ou contacter le support technique si vous êtes un client entreprise.\footnote{Liens : \url{https://neo4j.com/docs/}},\footnote{Liens : \url{https://neo4j.com/support/}} 
  \end{block}
\end{frame}

\begin{frame}{Communauté et événements}
  \begin{block}{Communauté Neo4j}
    Rejoignez la communauté en ligne de Neo4j pour partager des idées, poser des questions et collaborer sur des projets. Vous y trouverez des forums de discussion, des opportunités de collaboration et des événements communautaires.\footnote{Liens : \url{https://community.neo4j.com/}}\footnote{Liens : \url{https://community.neo4j.com/c/events/}} 
  \end{block}
  \begin{block}{Événements à venir}
    Découvrez les événements Neo4j à venir, tels que des conférences, des ateliers et des webinaires, pour approfondir vos connaissances et rencontrer d'autres passionnés de graphes.\footnote{Liens : \url{https://neo4j.com/events/}}
  \end{block}
\end{frame}
